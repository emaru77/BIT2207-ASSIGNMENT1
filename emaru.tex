\documentclass[10pt,a4paper]{report}
\usepackage[utf8]{inputenc}
\usepackage{amsmath}
\usepackage{amsfonts}
\usepackage{amssymb}
\begin{document}
•


\section{Introduction}
This section aims to describe the project background, problem statement, objectives, scopes, project significance and methodology of the proposed system. The system is an Events Ticketing System.
\section{Background for the study}
The development of the World Wide Web in the mid 1990’s opened up the commercial viability of the internet as for the first time, ordinary citizens were able to access the resources that it held. Years later the number of websites increased from tens of thousands to millions. The internet has changed the rules of trading by presenting fresh opportunities and altering the way firms engage in relationships with customers (Andam, 2003). The integration of information and communications technology in business has revolutionalised relationships between organizations and individuals. Specifically the use of ICT in business has enhanced productivity, encouraged greater customer participation and also reduced costs of doing business. it can also be used as an additional channel through which businesses communicate with and trade with customers (business to business ,B2C). 
Currently in Uganda most companies use manual systems to run their day to day tasks but these manual systems put pressure on people to be correct in all details of their work at all times and yet people aren’t perfect. With a manual system the level of service is dependent on individuals and this puts a requirement on management to run training continuously.
\section{statement}
Muk Entertainment currently uses printed tickets which clients purchase on arrival and make queues with their purchased tickets which are then showed to the entrance staff for verification before being allowed to enter and according to management this procedure is tiresome, time consuming and it is prone to errors as at times the customers inside the Event’s hall exceed the number of tickets printed thus the need for developing an automated events ticketing system.
\section{Objectives}
\subsection{Main objective}
To develop an automated Events ticketing system for Muk entertainment that will monitor and improve customers ‘access to their premises.
Specific objectives
To study and analyse the current ticketing system and procedures at Muk entertainment.
To design the proposed system.
To test and validate the proposed Events Ticketing System.
To implement and maintain the developed system.
\section{Scope}
\subsection{Subject scope}
This study will focus on the development of an Events ticketing for Muk Entertainment which will be online and web based. It will be used to issue out e-tickets to customers who will have made payments via this system.
\subsection{Geographical scope} 
Geographically the study will be carried out at the Muk Entertainment premises which are located next to the Muk swimming pool in the University. It will also involve two key stakeholders i.e. the management of Muk Entertainment and their esteemed clients.
\section{Significance of the study}
This system will highly contribute to the technological and economic value of Muk Entertainment in various ways which include;
This system will be cost effective as it will greatly reduce costs associated with printing and selling tickets to clients.
The system will improve the security and safety of customer tickets as these e-tickets will be verified using back code readers.
The management will k now the exact number of clients attending their events as this system will deliver actual attendance reports.
The system will greatly reduce inconveniences at the events’ entrance as customers will not have to make long queues any more.

\section{Conclusion}
There is a lot of fraud and risk taking involved in the ticketing industry in Uganda. This means event owners have difficulty pre-selling their tickets because of fears of duplication. Therefore implementing an automated Events Ticketing System could greatly reduce some of the highlighted problems.


\end{document}